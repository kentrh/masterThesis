\chapter{Mobile features}
\label{chapter_mobile_features}

Having a news application in mind, following are a description the most important aspects when creating a mobile application and how it is different from traditional desktop applications, considering the various features and abilities that are involved.

\section{Display}
The mobile displays are smaller than regular PC screens and the mobile user interfaces needs another approach than the PC UIs when designing them. News read on a desktop computer is mainly accessed through a web browser by visiting the content publishers' own websites or by other news aggregating services available on the Internet. Most news related native desktop applications available today are just RSS readers, which links to a web page, and the news article is still accessed via a web browser. This means that the designing of the news providing application is OS independent. Also most PC screens are large enough to display a web page with its content, not having to worry about the different screen sizes and adapting the content to every possible size, as when designing for mobile devices.

As of now, the iPhone only has two different screen sizes, where only the length of the screen differ, making the design process of adapting the application to both devices less comprehensive compared to other mobile devices. For instance, applications developed for mobile phones running on Android or Windows Phone, has a lot of different resolutions and screens sizes to take into account when designing the UI for the application.

Given that all information that should be displayed in a news web page fits inside a PC screen, it will most likely not fit inside a mobile screen. The design approach on a mobile UI will probably result in dividing the information up in smaller pieces and views, and some navigation logic between them has to be added.

\section{Interaction}
The interaction pattern for PCs and smart mobile devices are quite different. PCs are usually controlled with a keyboard and a mouse, while smart phones are normally controlled through a touch based screen.

On systems where a mouse is used, which accounts for most computers, the possibility of bringing up information when the mouse hovers over an element is widely used, but on mobile user interfaces with touch screens, the hover possibility is not an option, ergo another way of showing the same information has to be created.

On the other hand a touch device has a lot of possibilities that are not available or not as intuitive on a standard desktop computer. With a touch device, a user interacts directly with the element on a screen, compared to using a mouse. Gestures like dragging, pinching, swiping etc. can make a mobile user interface intuitive and easy to use, and more similar to gestures people do with actual objects. For instance, a swipe to the side to flip a page, similar to flipping a page in a real newspaper, is a more authentic experience to the real life than moving the mouse cursor to a page, clicking the mouse button and holding it down, and then drag it to flip the page.

Also a lot of mobile touch devices can register a lot of different event triggers, like touches, gestures, and movements, to mention a few. It is possible to differentiate between a touch with two fingers versus a touch with four fingers, a swipe with one finger versus a swipe with three fingers, and these gestures can also be combined. For instance, a user can touch and hold a box on the screen with two fingers for one second to trigger an edit mode, and further move this box around by starting to swipe two fingers while still holding on to the box, to position it wherever the user prefers, and then place the box by removing the fingers from the screen. Another possibility is to use a gyroscope, to have movement trigger some event in the mobile user interface. If a mobile phone is placed on a table facing up, and the phone starts ringing, turning the phone face down can trigger the silent function, for instance.

\section{Mobility}
An essential difference between PCs and mobile devices is the mobile device's mobility. A mobile smart phone is carried around with the user all day making it an easier accessible device, which again most likely makes the threshold for starting a device session is lower. Also the mobile device has become more personal than the personal computer, and it is probably safe to say that a mobile phone can be connected with one single user, while a PC more frequently are used by several users.

\subsection{User location}
Most mobile devices has one or more ways of collecting the device's location, either by a GPS, using the WiFi the device is connected to or using the base station the mobile network is connected to. When obtained, the device's location can be used to provide the user with news that resided nearby, either fresh news or historical news about events that happened a while back.

\subsection{Connectivity}
Mobile devices has the ability to stay connected to the Internet almost everywhere and all the time, making it quite effortless to check the latest news, and using the mobile device as a pastime activity while waiting for someone or riding the bus. In this manner, people are shifting towards a longer and more frequent mobile news consumption trend\cite{stateofthemedia2012}.

\subsection{Scanning}
Many mobile device has the ability to scan QR codes, RFID chips or other types of NFC units. If, for instance, a user walks by a poster advertising for a concert that is about to happen, and the poster has a QR code containing the information of the name of the artist, where it is playing etc. the mobile user can scan this code with its mobile device, and by some processing on the back-end side, bring up news about this happening.

\section{Software}
Most mobile devices can only run one application at a time, making the possibility to integrate features from other applications seamlessly an important aspect when designing and developing a mobile application. For instance, when using a mobile news application many users may feel the need to share a story they find interesting on a social network. This should be doable without costing the user too much effort, and without having the user to loose its news reading focus by having to perform a lot of operations to get it done. By integrating this possibility in an effortless way, sharing a news story to social network can be done in matter of a few clicks. 

All the most popular mobile operating systems, like iOS, Android and Windows Phone, allows for an easy integration between applications, to make sharing content to social networks an effortless task. In the mobile news application developed as part of this thesis an article can be shared to Facebook, Twitter or via mail by simply holding down one finger on the RSS perspective that are showing the news article to bring up the sharing control, and then choosing the desired sharing channel. Then an OS integrated sharing view is brought up and the user can share the story right from the news application without having to switch to another application.

This approach extends much further than just sharing content on social sites. If the external service the application wants to make use of, is not incorporated in the OS, the external service most likely has an SDK that can be included in the application to make use of the service directly from the application. This be showing a place on a map, sharing content, saving stories for reading later or watching movie clips.

\section{Monetizing}
As the trend of consuming news content is shifting from purchasing and reading real newspapers to accessing free news content on the Internet, news publishers has to use other approaches to earn money. There are several monetizing approaches to earn money through a news mobile application, for instance advertising, paid subscription, in-app purchases or paid applications.

To buy applications or performing other types of purchases on the different mobile application stores, the user usually has to store its credit card information on beforehand, making later purchases a quick and easy process. This can make the threshold for paying for content lower, and heighten the income.

\subsection{Advertising}
Advertising is a widely used approach on news web sites to earn money. The content providers can sell advertising space on their web sites directly to companies that want to advertise, or they can use a middle layer advertising service that provides the content publishers with ads, and in this way not having to get advertisers themselves. 

A problem with using ads on web sites on desktop computers is that there are several plug-ins for the various web browsers that can remove ad banners for the user. On the mobile platform this is not a problem big problem yet, as the browsers used on mobile phones not are as configurable as browsers on desktop computers.

There are a lot, probably several hundred, of different mobile advertising services, which offers different types of services and integrates nicely with a mobile application, the main issue is to find the right one.

\subsection{Paid Subscription}
Paid subscription is a way for the content publishers to give the paying customer more content than what is available to all the other non-paying users, and it also creates a form of a lock-in, as the subscriptions normally renews themselves and is automatically deducted from the users account. The news applications that offers a subscription are often just a digitalized version of the paper version of the same newspaper, and are available through the different mobile applications stores. These applications are also mainly provided by the news publishers themselves and not by third parties. The subscription based applications are also often a special type of application, available mainly for newspapers and magazines to digitally distribute their content.

There are also other ways of offering paid subscriptions, through web applications for instance, where a subscription can be bought through other payment services than the ones offered by the mobile application stores. This is an approach preferred by many content providers as the application stores often takes quite a large percentage of the revenue made through the app stores.

\subsection{In-app Purchases}
In-app purchases is a way of selling content or services inside a mobile application. As the credit card information normally already are stored, this is an easy way for the user to purchase content and services. A common strategy is to offer a free mobile application to make the threshold of downloading it lower, and further include in-app purchases to gain revenue inside the application. It is also a pretty trivial process for the developers and to include and execute an in-app purchase inside the application. An in-app purchase can, for instance, remove advertisement inside the application for a certain amount of money, for a certain amount of time, or reveal premium content not available to non-paying customers.

\subsection{Paid Applications}
Applications can also be sold on the application stores for a certain amount of money. This may be the least profitable approach, as users most likely want to try the application before paying. This can however be solved by offering a free application in addition with less functionality or content, so the user has the ability to try before buying.