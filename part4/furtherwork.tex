\chapter{Further Work}
\label{further_work}

Following are a proposal for further work to be carried out as a future part of this study.

\subsubsection{Perspectives}
A further study to see if there are other perspectives that are used in other applications, and if so, evaluating these to see how they are used and if they are suited for other purposes as well. A possible perspective that could be implemented, but not has been in any of the mobile news applications examined in this thesis, are a trend perspective, that shows a topic's or entity's popularity over time.


\subsubsection{Implementation}
The use case application presented in section \ref{use_case_application} were meant to include all of the perspectives describes in section \ref{perspectives_presentation} to see how it could be solved in navigational and presentational manner. 

As of now the application has the RSS perspective, full article perspective and map perspective, but how to implement the others as well has been thought of and should be implemented in future work. 

The event, entity and web perspective can be put at the same level as the full article perspective, to provide additional information, or another form of viewing the information gathered from the full article view. This could be done by swiping down in full article perspective to reveal one button for each of the other perspective and allowing the user to choose which type of perspective it wants to view the news article from.

The summary perspective could be placed between the RSS perspective and the full article perspective in the navigation hierarchy, to provide the user with an extra level, where more information about the article is shown, but not all of it, to get the notion of what the article is concerning before opening the full article.

\subsubsection{Testing}
Larger scaled and more comprehensive testing of the use case application and how the different perspectives are connected and presented in regard to user experience and usability, ought to be carried out. Also testing to see if perspectives used in a different manner, i.e. another navigation logic than the one that is most adapted by applications already developed, can be suited for users.


\subsubsection{Research}
A future research on why the different news applications are designed as they are and why there is an overweight of native third-party news applications, and not so much native news applications created by the content publishers themselves, could have been an interesting new depth to this study.